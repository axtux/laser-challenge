\documentclass[a4paper,10pt]{article}
\usepackage[utf8]{inputenc}

%opening
\title{
    \begin{minipage}\linewidth
        \centering\bfseries\sffamily
        Projet Génie Logiciel : LazerChallenge
        \vskip3pt
        \large Rapport d'implémentation
    \end{minipage}
    }
\author{Maazouz Mehdi, Lecocq Alexis}

\begin{document}

\maketitle
\tableofcontents
\newpage
\section{Introduction}
Nous débutons notre rapport d'implémentation par un léger rappel du projet qui nous a été demandé initialement. 
L'énoncé mis à notre disposition en début d'année parlait de la réalisation du jeu nommé ``Lazer Challenge'' en Java
et en nous aidant de la bibliothèque LibGDX. De plus , une extension était demandée individuellement à chaque étudiant .

Pour pouvoir réaliser ce projet, plusieurs étapes ont dû être nécessaires. La première consisait en la plannification de notre projet,
 cette étape comprenait la création d'un graphe Gantt , PERT, ainsi qu'une gestion des risques, une répartition du travail,
  et de pouvoir déterminer les ressources qui allaient noue être nécessaires dans l'optique de nous préparer au mieux à d'éventuel
 obstacles, empêchement qui pourrait subvenir et nous ralentir dans la conception de notre jeu. Ensuite est venue l'étape de la modélisation, où nous
 avons dû créer une maquette permettant aux professeurs et assistants d'avoir une vision de la forme que prendrai notre jeu , c'est à dire la partie graphique.
 Mais également la conception de plusieurs diagrammes UML , notamment un diagramme de classe qui a permis de montrer le fond de notre jeu , autrement dit les classes
 que comprendrai notre projet.
 
 Nous voici désormais à l'étape de l'implémentation où nous allont maintenant vous expliquer notre travail dans son ensemble,
 les difficultés et problèmes survenus , nos choix d'implémentations, ... TODO suite.
 
\section{Conception}
\subsection{Changement de groupe}
Dès le début de la phase d'implémentation , suite a plusieurs abandons dans les autres groupes, un membre de notre goupe a été annexé dans un autre groupe.
En conséquence , nous avons dû immédiatement revoir notre diagramme de Classe et la maquette graphique afin de pouvoir nous baser directement sur quelque chose
de concret. Le diagramme de classe se trouve dans le .jar et est donc à votre disposition (core/assets/images/ClassDiagram.png).
\subsection{Choix d'implémentation}
Nous allons ici , essayer d'argumenter de la meilleure manières nos choix que nous avons pris lors de la phase d'implémentation.
Un des choix les plus importants concerne les propriétés disponibles sur les tuiles. Au départ, nous avions respectés les consignes qui nous
avaient été données , mais après quelques soucis d'implémentation ( pas adapté à notre Design, quelques bugs également difficiles à régler ), nous
avont opté pour un mécanisme de restriction , autrement dit, chaque tuile possédera, ou non, une propriété appelé restriction. Cette dernière aura
comme valeur le nom d'un bloqueur, exemple , si une tuile possède restriction ayant comme valeur source, seul une source pourra être déposée sur cette tuile.
Si elle ne possède aucune restriction, tout type de bloqueurs pourront y être déposés.
\subsection{Difficultés}
Une des difficultés principale concernant le projet est l'appropriation de la bibliothèque LibGDX, ce qui est normal étant donné , qu'aucun des membres
n'avaient utilisés cette bibliothèque aupparavent , cependant nous regrettons un manque de documentation ...
\subsection{Concessions}
Suite à un manque de temps , nous avons dû revoir nos objectifs et avons priviliégié ce qui nous semble être le plus important. 
Ces objectifs ont été de pouvoir rendre un projet jouable, bien documenté , composés de quelques tests unitaires sur le côté , mais 
également un rapport fourni et une vidéo expliquant la manière de jouer au jeu et ainsi, d'en profiter pleinement. 
Vous l'avez donc sans doute remarqué, les extentions ne sont donc pas présente dans notre jeu final ... ou du moins , elles ne seront pas finalisées.
En effet , pour l'extension ``diagonal directions'' de Maazouz Mehdi, les maps avaient été créées, certains sprites spécifiques à ce mode de jeu également
et vous pourrze noter la présence de fichier xml spécifiques eux aussi, à cette extension . Bien entendu, cela n'a pas été suffisant pour la qualifer de finie.
Alexis Lecocq devait initialement, s'occuper de l'extension ``Saving + Multiple users + Social network''. Après réflexion, son choix
s'est tourné vers le réseau social Twitter .Suite à quelques problèmes, il est venu trouver les assistants et professeurs afin d'obtenir un hébergement,
malheureusement , il semblerait que ca n'est pas été le cas .


\end{document}
